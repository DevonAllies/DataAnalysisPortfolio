% Options for packages loaded elsewhere
% Options for packages loaded elsewhere
\PassOptionsToPackage{unicode}{hyperref}
\PassOptionsToPackage{hyphens}{url}
\PassOptionsToPackage{dvipsnames,svgnames,x11names}{xcolor}
%
\documentclass[
  letterpaper,
  DIV=11,
  numbers=noendperiod]{scrartcl}
\usepackage{xcolor}
\usepackage{amsmath,amssymb}
\setcounter{secnumdepth}{-\maxdimen} % remove section numbering
\usepackage{iftex}
\ifPDFTeX
  \usepackage[T1]{fontenc}
  \usepackage[utf8]{inputenc}
  \usepackage{textcomp} % provide euro and other symbols
\else % if luatex or xetex
  \usepackage{unicode-math} % this also loads fontspec
  \defaultfontfeatures{Scale=MatchLowercase}
  \defaultfontfeatures[\rmfamily]{Ligatures=TeX,Scale=1}
\fi
\usepackage{lmodern}
\ifPDFTeX\else
  % xetex/luatex font selection
\fi
% Use upquote if available, for straight quotes in verbatim environments
\IfFileExists{upquote.sty}{\usepackage{upquote}}{}
\IfFileExists{microtype.sty}{% use microtype if available
  \usepackage[]{microtype}
  \UseMicrotypeSet[protrusion]{basicmath} % disable protrusion for tt fonts
}{}
\makeatletter
\@ifundefined{KOMAClassName}{% if non-KOMA class
  \IfFileExists{parskip.sty}{%
    \usepackage{parskip}
  }{% else
    \setlength{\parindent}{0pt}
    \setlength{\parskip}{6pt plus 2pt minus 1pt}}
}{% if KOMA class
  \KOMAoptions{parskip=half}}
\makeatother
% Make \paragraph and \subparagraph free-standing
\makeatletter
\ifx\paragraph\undefined\else
  \let\oldparagraph\paragraph
  \renewcommand{\paragraph}{
    \@ifstar
      \xxxParagraphStar
      \xxxParagraphNoStar
  }
  \newcommand{\xxxParagraphStar}[1]{\oldparagraph*{#1}\mbox{}}
  \newcommand{\xxxParagraphNoStar}[1]{\oldparagraph{#1}\mbox{}}
\fi
\ifx\subparagraph\undefined\else
  \let\oldsubparagraph\subparagraph
  \renewcommand{\subparagraph}{
    \@ifstar
      \xxxSubParagraphStar
      \xxxSubParagraphNoStar
  }
  \newcommand{\xxxSubParagraphStar}[1]{\oldsubparagraph*{#1}\mbox{}}
  \newcommand{\xxxSubParagraphNoStar}[1]{\oldsubparagraph{#1}\mbox{}}
\fi
\makeatother

\usepackage{color}
\usepackage{fancyvrb}
\newcommand{\VerbBar}{|}
\newcommand{\VERB}{\Verb[commandchars=\\\{\}]}
\DefineVerbatimEnvironment{Highlighting}{Verbatim}{commandchars=\\\{\}}
% Add ',fontsize=\small' for more characters per line
\usepackage{framed}
\definecolor{shadecolor}{RGB}{241,243,245}
\newenvironment{Shaded}{\begin{snugshade}}{\end{snugshade}}
\newcommand{\AlertTok}[1]{\textcolor[rgb]{0.68,0.00,0.00}{#1}}
\newcommand{\AnnotationTok}[1]{\textcolor[rgb]{0.37,0.37,0.37}{#1}}
\newcommand{\AttributeTok}[1]{\textcolor[rgb]{0.40,0.45,0.13}{#1}}
\newcommand{\BaseNTok}[1]{\textcolor[rgb]{0.68,0.00,0.00}{#1}}
\newcommand{\BuiltInTok}[1]{\textcolor[rgb]{0.00,0.23,0.31}{#1}}
\newcommand{\CharTok}[1]{\textcolor[rgb]{0.13,0.47,0.30}{#1}}
\newcommand{\CommentTok}[1]{\textcolor[rgb]{0.37,0.37,0.37}{#1}}
\newcommand{\CommentVarTok}[1]{\textcolor[rgb]{0.37,0.37,0.37}{\textit{#1}}}
\newcommand{\ConstantTok}[1]{\textcolor[rgb]{0.56,0.35,0.01}{#1}}
\newcommand{\ControlFlowTok}[1]{\textcolor[rgb]{0.00,0.23,0.31}{\textbf{#1}}}
\newcommand{\DataTypeTok}[1]{\textcolor[rgb]{0.68,0.00,0.00}{#1}}
\newcommand{\DecValTok}[1]{\textcolor[rgb]{0.68,0.00,0.00}{#1}}
\newcommand{\DocumentationTok}[1]{\textcolor[rgb]{0.37,0.37,0.37}{\textit{#1}}}
\newcommand{\ErrorTok}[1]{\textcolor[rgb]{0.68,0.00,0.00}{#1}}
\newcommand{\ExtensionTok}[1]{\textcolor[rgb]{0.00,0.23,0.31}{#1}}
\newcommand{\FloatTok}[1]{\textcolor[rgb]{0.68,0.00,0.00}{#1}}
\newcommand{\FunctionTok}[1]{\textcolor[rgb]{0.28,0.35,0.67}{#1}}
\newcommand{\ImportTok}[1]{\textcolor[rgb]{0.00,0.46,0.62}{#1}}
\newcommand{\InformationTok}[1]{\textcolor[rgb]{0.37,0.37,0.37}{#1}}
\newcommand{\KeywordTok}[1]{\textcolor[rgb]{0.00,0.23,0.31}{\textbf{#1}}}
\newcommand{\NormalTok}[1]{\textcolor[rgb]{0.00,0.23,0.31}{#1}}
\newcommand{\OperatorTok}[1]{\textcolor[rgb]{0.37,0.37,0.37}{#1}}
\newcommand{\OtherTok}[1]{\textcolor[rgb]{0.00,0.23,0.31}{#1}}
\newcommand{\PreprocessorTok}[1]{\textcolor[rgb]{0.68,0.00,0.00}{#1}}
\newcommand{\RegionMarkerTok}[1]{\textcolor[rgb]{0.00,0.23,0.31}{#1}}
\newcommand{\SpecialCharTok}[1]{\textcolor[rgb]{0.37,0.37,0.37}{#1}}
\newcommand{\SpecialStringTok}[1]{\textcolor[rgb]{0.13,0.47,0.30}{#1}}
\newcommand{\StringTok}[1]{\textcolor[rgb]{0.13,0.47,0.30}{#1}}
\newcommand{\VariableTok}[1]{\textcolor[rgb]{0.07,0.07,0.07}{#1}}
\newcommand{\VerbatimStringTok}[1]{\textcolor[rgb]{0.13,0.47,0.30}{#1}}
\newcommand{\WarningTok}[1]{\textcolor[rgb]{0.37,0.37,0.37}{\textit{#1}}}

\usepackage{longtable,booktabs,array}
\usepackage{calc} % for calculating minipage widths
% Correct order of tables after \paragraph or \subparagraph
\usepackage{etoolbox}
\makeatletter
\patchcmd\longtable{\par}{\if@noskipsec\mbox{}\fi\par}{}{}
\makeatother
% Allow footnotes in longtable head/foot
\IfFileExists{footnotehyper.sty}{\usepackage{footnotehyper}}{\usepackage{footnote}}
\makesavenoteenv{longtable}
\usepackage{graphicx}
\makeatletter
\newsavebox\pandoc@box
\newcommand*\pandocbounded[1]{% scales image to fit in text height/width
  \sbox\pandoc@box{#1}%
  \Gscale@div\@tempa{\textheight}{\dimexpr\ht\pandoc@box+\dp\pandoc@box\relax}%
  \Gscale@div\@tempb{\linewidth}{\wd\pandoc@box}%
  \ifdim\@tempb\p@<\@tempa\p@\let\@tempa\@tempb\fi% select the smaller of both
  \ifdim\@tempa\p@<\p@\scalebox{\@tempa}{\usebox\pandoc@box}%
  \else\usebox{\pandoc@box}%
  \fi%
}
% Set default figure placement to htbp
\def\fps@figure{htbp}
\makeatother


% definitions for citeproc citations
\NewDocumentCommand\citeproctext{}{}
\NewDocumentCommand\citeproc{mm}{%
  \begingroup\def\citeproctext{#2}\cite{#1}\endgroup}
\makeatletter
 % allow citations to break across lines
 \let\@cite@ofmt\@firstofone
 % avoid brackets around text for \cite:
 \def\@biblabel#1{}
 \def\@cite#1#2{{#1\if@tempswa , #2\fi}}
\makeatother
\newlength{\cslhangindent}
\setlength{\cslhangindent}{1.5em}
\newlength{\csllabelwidth}
\setlength{\csllabelwidth}{3em}
\newenvironment{CSLReferences}[2] % #1 hanging-indent, #2 entry-spacing
 {\begin{list}{}{%
  \setlength{\itemindent}{0pt}
  \setlength{\leftmargin}{0pt}
  \setlength{\parsep}{0pt}
  % turn on hanging indent if param 1 is 1
  \ifodd #1
   \setlength{\leftmargin}{\cslhangindent}
   \setlength{\itemindent}{-1\cslhangindent}
  \fi
  % set entry spacing
  \setlength{\itemsep}{#2\baselineskip}}}
 {\end{list}}
\usepackage{calc}
\newcommand{\CSLBlock}[1]{\hfill\break\parbox[t]{\linewidth}{\strut\ignorespaces#1\strut}}
\newcommand{\CSLLeftMargin}[1]{\parbox[t]{\csllabelwidth}{\strut#1\strut}}
\newcommand{\CSLRightInline}[1]{\parbox[t]{\linewidth - \csllabelwidth}{\strut#1\strut}}
\newcommand{\CSLIndent}[1]{\hspace{\cslhangindent}#1}



\setlength{\emergencystretch}{3em} % prevent overfull lines

\providecommand{\tightlist}{%
  \setlength{\itemsep}{0pt}\setlength{\parskip}{0pt}}



 


\KOMAoption{captions}{tableheading}
\makeatletter
\@ifpackageloaded{caption}{}{\usepackage{caption}}
\AtBeginDocument{%
\ifdefined\contentsname
  \renewcommand*\contentsname{Table of contents}
\else
  \newcommand\contentsname{Table of contents}
\fi
\ifdefined\listfigurename
  \renewcommand*\listfigurename{List of Figures}
\else
  \newcommand\listfigurename{List of Figures}
\fi
\ifdefined\listtablename
  \renewcommand*\listtablename{List of Tables}
\else
  \newcommand\listtablename{List of Tables}
\fi
\ifdefined\figurename
  \renewcommand*\figurename{Figure}
\else
  \newcommand\figurename{Figure}
\fi
\ifdefined\tablename
  \renewcommand*\tablename{Table}
\else
  \newcommand\tablename{Table}
\fi
}
\@ifpackageloaded{float}{}{\usepackage{float}}
\floatstyle{ruled}
\@ifundefined{c@chapter}{\newfloat{codelisting}{h}{lop}}{\newfloat{codelisting}{h}{lop}[chapter]}
\floatname{codelisting}{Listing}
\newcommand*\listoflistings{\listof{codelisting}{List of Listings}}
\makeatother
\makeatletter
\makeatother
\makeatletter
\@ifpackageloaded{caption}{}{\usepackage{caption}}
\@ifpackageloaded{subcaption}{}{\usepackage{subcaption}}
\makeatother
\usepackage{bookmark}
\IfFileExists{xurl.sty}{\usepackage{xurl}}{} % add URL line breaks if available
\urlstyle{same}
\hypersetup{
  pdftitle={Pathogen Identification from Clinical Metagenomics Data},
  pdfauthor={Devon Allies},
  colorlinks=true,
  linkcolor={blue},
  filecolor={Maroon},
  citecolor={Blue},
  urlcolor={Blue},
  pdfcreator={LaTeX via pandoc}}


\title{Pathogen Identification from Clinical Metagenomics Data}
\author{Devon Allies}
\date{}
\begin{document}
\maketitle
\begin{abstract}
This analysis focuses on identifying a potential pathogen from clinical
metagenomics data. The workflow includes quality control of sequencing
reads, host depletion, de novo assembly, and classification using BLAST.
The results highlight the presence of clinically relevant pathogens in
the sample.
\end{abstract}


\subsection{Aim:}\label{aim}

The objective of this project was to develop and execute a
bioinformatics workflow to identify potential pathogens from clinical
metagenomics data. The workflow encompassed several key steps, including
quality control of raw sequencing reads, host depletion to remove human
DNA, de novo assembly of unmapped reads, and classification using BLAST.
The ultimate goal was to determine the presence of clinically relevant
pathogens in the sample. By performing host depletation and
classification, I was able to demonstrate the ability to isolate the
microbial signal from a Bronchoalveolar lavage (BAL) sample.

\subsection{Data Acquisition:}\label{data-acquisition}

The raw sequencing data for this project was accessed from the NCBI SRA
(BioProject: PRJNA1327646), originating from a published study on COPD
microbiomes (Chen et al., 2025). The authors of the study specified that
the data deposited in the SRA had already undergone in silico human host
DNA removal prior to submission.

To ensure data integrity and quantify any residual human reads, a
Quality Control (QC) step was performed. Raw reads were aligned against
the human reference genome (hg38) using BWA (Li and Durbin, 2009).

\subsection{Quality Control of
Samples:}\label{quality-control-of-samples}

\subsubsection{Fastq Statistics:}\label{fastq-statistics}

The \texttt{seqkit\ stats} (Shen, Sipos and Zhao, 2024) output describes
the raw sequencing data before alignment. The command is used to
generate a tabular statistical summary of \texttt{FASTA/Q} files. By
default, it automatically detects the sequence type of the files.

\begin{table}

\caption{\label{tbl-fq-stats}Statistics of fastq files}

\centering{

\begin{Shaded}
\begin{Highlighting}[]
\VariableTok{envDir}\OperatorTok{=}\StringTok{"/home/devonallies/micromamba/envs/bioinfo/bin"}
\VariableTok{dataDir}\OperatorTok{=}\StringTok{"../../../../data"}

\VariableTok{$\{envDir\}}\ExtensionTok{/seqkit}\NormalTok{ stats }\VariableTok{$\{dataDir\}}\NormalTok{/reads/SRR35359600\_}\PreprocessorTok{*}\NormalTok{.fastq}
\end{Highlighting}
\end{Shaded}

\begin{verbatim}
processed files:  0 / 2 [------] ETA: 0s
processed files:  2 / 2 [======] ETA: 0s
processed files:  2 / 2 [] ETA: 0s. done
processed files:  2 / 2 [] ETA: 0s. done
file                                        format  type   num_seqs      sum_len  min_len  avg_len  max_len
../../../../data/reads/SRR35359600_1.fastq  FASTQ   DNA   1,246,647  182,039,361       15      146      150
../../../../data/reads/SRR35359600_2.fastq  FASTQ   DNA   1,246,647  181,919,031       15    145.9      150
\end{verbatim}

}

\end{table}%

\subsection{BAM Flagstat Statistics:}\label{bam-flagstat-statistics}

The \texttt{BAM\ flagstat} statistic summarizes the results after the
raw reads have been aligned / mapped to a reference genome.

\begin{table}

\caption{\label{tbl-bam-flagstat}Statistics of BAM files}

\centering{

\begin{Shaded}
\begin{Highlighting}[]
\VariableTok{envDir}\OperatorTok{=}\StringTok{"/home/devonallies/micromamba/envs/bioinfo/bin"}
\VariableTok{dataDir}\OperatorTok{=}\StringTok{"../../../../data"}
\VariableTok{sample}\OperatorTok{=}\VariableTok{$(}\FunctionTok{basename} \StringTok{"}\VariableTok{$\{dataDir\}}\StringTok{/sort/SRR35359600.sorted.bam"}\NormalTok{ .sorted.bam}\VariableTok{)}

\BuiltInTok{echo} \StringTok{"{-}{-}{-}{-}{-}{-}{-}{-}{-}{-}{-}{-}{-}{-}{-}{-}{-}{-}{-}{-}{-}{-}{-}{-}{-}{-}{-}{-}{-}{-}"}
\BuiltInTok{echo} \StringTok{"Processing sample: }\VariableTok{$sample}\StringTok{"}
\BuiltInTok{echo} \StringTok{"{-}{-}{-}{-}{-}{-}{-}{-}{-}{-}{-}{-}{-}{-}{-}{-}{-}{-}{-}{-}{-}{-}{-}{-}{-}{-}{-}{-}{-}{-}{-}"}
\VariableTok{$\{envDir\}}\ExtensionTok{/samtools}\NormalTok{ flagstat }\VariableTok{$\{dataDir\}}\NormalTok{/sort/SRR35359600.sorted.bam}
\end{Highlighting}
\end{Shaded}

\begin{verbatim}
------------------------------
Processing sample: SRR35359600
-------------------------------
2627660 + 0 in total (QC-passed reads + QC-failed reads)
2493294 + 0 primary
0 + 0 secondary
134366 + 0 supplementary
0 + 0 duplicates
0 + 0 primary duplicates
1637803 + 0 mapped (62.33% : N/A)
1503437 + 0 primary mapped (60.30% : N/A)
2493294 + 0 paired in sequencing
1246647 + 0 read1
1246647 + 0 read2
1169744 + 0 properly paired (46.92% : N/A)
1496222 + 0 with itself and mate mapped
7215 + 0 singletons (0.29% : N/A)
241104 + 0 with mate mapped to a different chr
113350 + 0 with mate mapped to a different chr (mapQ>=5)
\end{verbatim}

}

\end{table}%

\begin{itemize}
\item
  \texttt{total}: The total number of reads processed, separated into
  those passing Quality Control (QC) and those failing.
\item
  \texttt{mapped}: The percentage of reads that successfully aligned to
  the reference genome. This is critical for assessing the efficiency of
  the alignment process and the relevance of the reference genome.
\item
  \texttt{paired\ in\ sequencing}: Confirms all reads were generated
  from paired-end sequencing, which is important for downstream analyses
  that rely on paired-end data.
\item
  \texttt{properly\ paired}: Reads that aligned to the reference genome
  in the expected orientation and distance. A high percentage indicates
  good library preparation and alignment quality.
\item
  \texttt{singletons}: Reads that did not have a properly paired mate.
\end{itemize}

\subsection{Host Depletion:}\label{host-depletion}

Host depletion is a critical step in viral metagenomics to enrich for
viral sequences by removing host-derived reads. This process can be
achieved through various methods, including computational approaches
that filter out host sequences based on alignment to a reference genome
or using specialized tools designed for host depletion.

\begin{table}

\caption{\label{tbl-host-depletion}Removing unmapped reads}

\centering{

\begin{Shaded}
\begin{Highlighting}[]
\CommentTok{\# Set the paths for files and environment}
\VariableTok{envDir}\OperatorTok{=}\StringTok{"/home/devonallies/micromamba/envs/bioinfo/bin"}
\VariableTok{dataDir}\OperatorTok{=}\StringTok{"../../../../data"}

\VariableTok{sample}\OperatorTok{=}\VariableTok{$(}\FunctionTok{basename} \StringTok{"}\VariableTok{$\{dataDir\}}\StringTok{/sort/SRR35359600.sorted.bam"}\NormalTok{ .sorted.bam}\VariableTok{)}

\BuiltInTok{echo} \StringTok{"{-}{-}{-}{-}{-}{-}{-}{-}{-}{-}{-}{-}{-}{-}{-}{-}{-}{-}{-}{-}{-}{-}{-}{-}{-}{-}{-}{-}{-}{-}"}
\BuiltInTok{echo} \StringTok{"Processing sample: }\VariableTok{$sample}\StringTok{"}
\BuiltInTok{echo} \StringTok{"{-}{-}{-}{-}{-}{-}{-}{-}{-}{-}{-}{-}{-}{-}{-}{-}{-}{-}{-}{-}{-}{-}{-}{-}{-}{-}{-}{-}{-}{-}{-}"}

\VariableTok{$\{envDir\}}\ExtensionTok{/samtools}\NormalTok{ collate }\AttributeTok{{-}u} \AttributeTok{{-}O} \VariableTok{$\{dataDir\}}\NormalTok{/sort/SRR35359600.sorted.bam }\KeywordTok{|} \DataTypeTok{\textbackslash{}}
\VariableTok{$\{envDir\}}\ExtensionTok{/samtools}\NormalTok{ fastq }\AttributeTok{{-}f4} \AttributeTok{{-}1} \StringTok{"}\VariableTok{$\{dataDir\}}\StringTok{/unmapped/}\VariableTok{$\{sample\}}\StringTok{\_R1.fq"} \AttributeTok{{-}2} \StringTok{"}\VariableTok{$\{dataDir\}}\StringTok{/unmapped/}\VariableTok{$\{sample\}}\StringTok{\_R2.fq"} \AttributeTok{{-}s} \StringTok{"}\VariableTok{$\{dataDir\}}\StringTok{/unmapped/}\VariableTok{$\{sample\}}\StringTok{\_singleton.fq"}
\end{Highlighting}
\end{Shaded}

\begin{verbatim}
------------------------------
Processing sample: SRR35359600
-------------------------------
[M::bam2fq_mainloop] discarded 7215 singletons
[M::bam2fq_mainloop] processed 989857 reads
\end{verbatim}

}

\end{table}%

Host depletion is a critical step in clinical metagenomics because
clinical samples (like blood, tissue, or respirator fluid) are typically
dominated by human DNA, which can constitute 90\% - 99\% of the
sequencing data. By removing the host background, sensitivity is
increased, it enables accurate De Novo Assembly, and improves downstream
analysis. The singletons are placed in a \texttt{singleton.fq} file.

While a notable percentage indicated in Table~\ref{tbl-bam-flagstat},
62.33\%, of the reads still aligned to the human genome, the remaining
non-host reads are sufficient for downstream classification.

\subsection{De Nova Assembly:}\label{de-nova-assembly}

Megahit (Li et al., 2016) is a fast and memory-efficient de novo
assembler for large and complex metagenomics data sets. It uses succinct
de Bruijn graphs to represent the input data, which significantly
reduces memory usage compared to traditional de Bruijn graph
representations. Megahit is designed to handle large-scale metagenomic
datasets, making it suitable for assembling genomes from environmental
samples with high diversity and complexity.

\begin{table}

\caption{\label{tbl-denova}De Nova Assembly uisng Megahits}

\centering{

\begin{Shaded}
\begin{Highlighting}[]
\CommentTok{\# Set the paths for files and environment}
\VariableTok{envDir}\OperatorTok{=}\StringTok{"/home/devonallies/micromamba/envs/bioinfo/bin"}
\VariableTok{dataDir}\OperatorTok{=}\StringTok{"../../../../data"}

\VariableTok{sample}\OperatorTok{=}\VariableTok{$(}\FunctionTok{basename} \StringTok{"}\VariableTok{$\{dataDir\}}\StringTok{/unmapped/SRR35359600"}\NormalTok{ \_R1.fq}\VariableTok{)}
\VariableTok{r1}\OperatorTok{=}\VariableTok{$\{dataDir\}}\NormalTok{/unmapped/}\VariableTok{$\{sample\}}\NormalTok{\_R1.fq}
\VariableTok{r2}\OperatorTok{=}\VariableTok{$\{dataDir\}}\NormalTok{/unmapped/}\VariableTok{$\{sample\}}\NormalTok{\_R2.fq}
\VariableTok{singleton}\OperatorTok{=}\VariableTok{$\{dataDir\}}\NormalTok{/unmapped/}\VariableTok{$\{sample\}}\NormalTok{\_singleton.fq}

\BuiltInTok{echo} \StringTok{"{-}{-}{-}{-}{-}{-}{-}{-}{-}{-}{-}{-}{-}{-}{-}{-}{-}{-}{-}{-}{-}{-}{-}{-}{-}{-}{-}{-}{-}{-}"}
\BuiltInTok{echo} \StringTok{"Processing sample: }\VariableTok{$sample}\StringTok{"}
\BuiltInTok{echo} \StringTok{"{-}{-}{-}{-}{-}{-}{-}{-}{-}{-}{-}{-}{-}{-}{-}{-}{-}{-}{-}{-}{-}{-}{-}{-}{-}{-}{-}{-}{-}{-}{-}"}

\VariableTok{$\{envDir\}}\ExtensionTok{/megahit} \AttributeTok{{-}1} \VariableTok{$r1} \AttributeTok{{-}2} \VariableTok{$r2} \AttributeTok{{-}r} \VariableTok{$singleton} \AttributeTok{{-}o} \StringTok{"}\VariableTok{$\{dataDir\}}\StringTok{/megahit/}\VariableTok{$\{sample\}}\StringTok{"} \AttributeTok{{-}{-}out{-}prefix} \VariableTok{$sample} \AttributeTok{{-}f}
\end{Highlighting}
\end{Shaded}

\begin{verbatim}
------------------------------
Processing sample: SRR35359600
-------------------------------
2026-02-06 13:14:03 - MEGAHIT v1.2.9
2026-02-06 13:14:03 - Using megahit_core with POPCNT and BMI2 support
2026-02-06 13:14:03 - Convert reads to binary library
2026-02-06 13:14:04 - b'INFO  sequence/io/sequence_lib.cpp  :   75 - Lib 0 (/mnt/069ABB979ABB822B/learning/data/unmapped/SRR35359600_R1.fq,/mnt/069ABB979ABB822B/learning/data/unmapped/SRR35359600_R2.fq): pe, 982642 reads, 150 max length'
2026-02-06 13:14:04 - b'INFO  sequence/io/sequence_lib.cpp  :   75 - Lib 1 (/mnt/069ABB979ABB822B/learning/data/unmapped/SRR35359600_singleton.fq): se, 7215 reads, 150 max length'
2026-02-06 13:14:04 - b'INFO  utils/utils.h                 :  152 - Real: 1.0586\tuser: 0.6510\tsys: 0.1466\tmaxrss: 85140'
2026-02-06 13:14:04 - k-max reset to: 141 
2026-02-06 13:14:04 - Start assembly. Number of CPU threads 8 
2026-02-06 13:14:04 - k list: 21,29,39,59,79,99,119,141 
2026-02-06 13:14:04 - Memory used: 45253907251
2026-02-06 13:14:04 - Extract solid (k+1)-mers for k = 21 
2026-02-06 13:14:13 - Build graph for k = 21 
2026-02-06 13:14:16 - Assemble contigs from SdBG for k = 21
2026-02-06 13:14:24 - Local assembly for k = 21
2026-02-06 13:14:30 - Extract iterative edges from k = 21 to 29 
2026-02-06 13:14:31 - Build graph for k = 29 
2026-02-06 13:14:33 - Assemble contigs from SdBG for k = 29
2026-02-06 13:14:40 - Local assembly for k = 29
2026-02-06 13:14:46 - Extract iterative edges from k = 29 to 39 
2026-02-06 13:14:46 - Build graph for k = 39 
2026-02-06 13:14:48 - Assemble contigs from SdBG for k = 39
2026-02-06 13:14:57 - Local assembly for k = 39
2026-02-06 13:15:05 - Extract iterative edges from k = 39 to 59 
2026-02-06 13:15:06 - Build graph for k = 59 
2026-02-06 13:15:07 - Assemble contigs from SdBG for k = 59
2026-02-06 13:15:13 - Local assembly for k = 59
2026-02-06 13:15:21 - Extract iterative edges from k = 59 to 79 
2026-02-06 13:15:22 - Build graph for k = 79 
2026-02-06 13:15:22 - Assemble contigs from SdBG for k = 79
2026-02-06 13:15:27 - Local assembly for k = 79
2026-02-06 13:15:34 - Extract iterative edges from k = 79 to 99 
2026-02-06 13:15:35 - Build graph for k = 99 
2026-02-06 13:15:35 - Assemble contigs from SdBG for k = 99
2026-02-06 13:15:39 - Local assembly for k = 99
2026-02-06 13:15:46 - Extract iterative edges from k = 99 to 119 
2026-02-06 13:15:46 - Build graph for k = 119 
2026-02-06 13:15:47 - Assemble contigs from SdBG for k = 119
2026-02-06 13:15:50 - Local assembly for k = 119
2026-02-06 13:15:57 - Extract iterative edges from k = 119 to 141 
2026-02-06 13:15:57 - Build graph for k = 141 
2026-02-06 13:15:58 - Assemble contigs from SdBG for k = 141
2026-02-06 13:16:01 - Merging to output final contigs 
2026-02-06 13:16:01 - 9116 contigs, total 7205375 bp, min 210 bp, max 16159 bp, avg 790 bp, N50 908 bp
2026-02-06 13:16:01 - ALL DONE. Time elapsed: 117.557580 seconds 
\end{verbatim}

}

\end{table}%

\begin{table}

\caption{\label{tbl-megahit-stats}Statistics of Megahit Assembly}

\centering{

\begin{Shaded}
\begin{Highlighting}[]
\CommentTok{\# Set the paths for files and environment}
\VariableTok{envDir}\OperatorTok{=}\StringTok{"/home/devonallies/micromamba/envs/bioinfo/bin"}
\VariableTok{dataDir}\OperatorTok{=}\StringTok{"../../../../data"}
\VariableTok{sample}\OperatorTok{=}\VariableTok{$(}\FunctionTok{basename} \StringTok{"}\VariableTok{$\{dataDir\}}\StringTok{/unmapped/SRR35359600"}\NormalTok{ \_R1.fq}\VariableTok{)}

\VariableTok{$\{envDir\}}\ExtensionTok{/seqkit}\NormalTok{ stats }\VariableTok{$\{dataDir\}}\NormalTok{/megahit/}\VariableTok{$\{sample\}}\NormalTok{/}\VariableTok{$\{sample\}}\NormalTok{.contigs.fa}
\end{Highlighting}
\end{Shaded}

\begin{verbatim}
file                                                         format  type  num_seqs    sum_len  min_len  avg_len  max_len
../../../../data/megahit/SRR35359600/SRR35359600.contigs.fa  FASTA   DNA      9,116  7,205,375      210    790.4   16,159
\end{verbatim}

}

\end{table}%

\subsection{Basic Local Alignment Search
Tool:}\label{basic-local-alignment-search-tool}

Blast is the gold standard for comparing a query sequence (DNA, RNA, or
Protein) against a massive database of known sequences to find regions
of similarity

\begin{Shaded}
\begin{Highlighting}[]
\CommentTok{\#|label: tbl{-}blast}
\CommentTok{\#|tbl{-}cap: "Output of blast result"}

\VariableTok{envDir}\OperatorTok{=}\StringTok{"/home/devonallies/micromamba/envs/bioinfo/bin"}
\VariableTok{dataDir}\OperatorTok{=}\StringTok{"../../../../data"}
\VariableTok{sample}\OperatorTok{=}\VariableTok{$(}\FunctionTok{basename} \StringTok{"}\VariableTok{$\{dataDir\}}\StringTok{/unmapped/SRR35359600"}\NormalTok{ \_R1.fq}\VariableTok{)}

\VariableTok{$\{envDir\}}\ExtensionTok{/makeblastdb} \AttributeTok{{-}out} \VariableTok{$\{dataDir\}}\NormalTok{/megahit/}\VariableTok{$\{sample\}}\NormalTok{/}\VariableTok{$\{sample\}} \AttributeTok{{-}in} \VariableTok{$\{dataDir\}}\NormalTok{/megahit/}\VariableTok{$\{sample\}}\NormalTok{/}\VariableTok{$\{sample\}}\NormalTok{.contigs.fa }\AttributeTok{{-}dbtype}\NormalTok{ nucl }\AttributeTok{{-}parse\_seqids}

\VariableTok{$\{envDir\}}\ExtensionTok{/blastdbcmd} \AttributeTok{{-}db} \VariableTok{$\{dataDir\}}\NormalTok{/megahit/}\VariableTok{$\{sample\}}\NormalTok{/}\VariableTok{$\{sample\}} \AttributeTok{{-}entry}\NormalTok{ all }\AttributeTok{{-}outfmt} \StringTok{"\%l \%a"} \OperatorTok{\textgreater{}} \StringTok{"}\VariableTok{$\{dataDir\}}\StringTok{/megahit/}\VariableTok{$\{sample\}}\StringTok{/}\VariableTok{$\{sample\}}\StringTok{\_blast.txt"}

\FunctionTok{cat} \VariableTok{$\{dataDir\}}\NormalTok{/megahit/}\VariableTok{$\{sample\}}\NormalTok{/}\VariableTok{$\{sample\}}\NormalTok{\_blast.txt }\KeywordTok{|} \FunctionTok{sort} \AttributeTok{{-}rn} \KeywordTok{|} \FunctionTok{head}
\end{Highlighting}
\end{Shaded}

\begin{verbatim}


Building a new DB, current time: 02/06/2026 13:16:01
New DB name:   /mnt/069ABB979ABB822B/learning/data/megahit/SRR35359600/SRR35359600
New DB title:  ../../../../data/megahit/SRR35359600/SRR35359600.contigs.fa
Sequence type: Nucleotide
Deleted existing Nucleotide BLAST database named /mnt/069ABB979ABB822B/learning/data/megahit/SRR35359600/SRR35359600
Keep MBits: T
Maximum file size: 3000000000B
Adding sequences from FASTA; added 9116 sequences in 0.687214 seconds.


16159 k141_9106
15399 k141_6088
15193 k141_7641
9747 k141_173
8971 k141_6816
8401 k141_4799
8326 k141_91
8324 k141_2364
7369 k141_8249
6914 k141_3132
\end{verbatim}

\subsection{Sample Identification:}\label{sample-identification}

\begin{table}

\caption{\label{tbl-idsample}Top BLAST Hits for Sample}

\centering{

\begin{Shaded}
\begin{Highlighting}[]
\VariableTok{envDir}\OperatorTok{=}\StringTok{"/home/devonallies/micromamba/envs/bioinfo/bin"}
\VariableTok{dataDir}\OperatorTok{=}\StringTok{"../../../../data"}
\VariableTok{sample}\OperatorTok{=}\VariableTok{$(}\FunctionTok{basename} \StringTok{"}\VariableTok{$\{dataDir\}}\StringTok{/unmapped/SRR35359600"}\NormalTok{ \_R1.fq}\VariableTok{)}

\VariableTok{$\{envDir\}}\ExtensionTok{/blastdbcmd} \AttributeTok{{-}db} \VariableTok{$\{dataDir\}}\NormalTok{/megahit/}\VariableTok{$\{sample\}}\NormalTok{/}\VariableTok{$\{sample\}} \AttributeTok{{-}entry}\NormalTok{ k141\_9107 }\OperatorTok{\textgreater{}} \StringTok{"}\VariableTok{$\{dataDir\}}\StringTok{/megahit/}\VariableTok{$\{sample\}}\StringTok{/k141\_9107.fa"}
\VariableTok{$\{envDir\}}\ExtensionTok{/blastdbcmd} \AttributeTok{{-}db} \VariableTok{$\{dataDir\}}\NormalTok{/megahit/}\VariableTok{$\{sample\}}\NormalTok{/}\VariableTok{$\{sample\}} \AttributeTok{{-}entry}\NormalTok{ k141\_6789 }\OperatorTok{\textgreater{}} \StringTok{"}\VariableTok{$\{dataDir\}}\StringTok{/megahit/}\VariableTok{$\{sample\}}\StringTok{/k141\_6789.fa"}
\VariableTok{$\{envDir\}}\ExtensionTok{/blastdbcmd} \AttributeTok{{-}db} \VariableTok{$\{dataDir\}}\NormalTok{/megahit/}\VariableTok{$\{sample\}}\NormalTok{/}\VariableTok{$\{sample\}} \AttributeTok{{-}entry}\NormalTok{ k141\_7744 }\OperatorTok{\textgreater{}} \StringTok{"}\VariableTok{$\{dataDir\}}\StringTok{/megahit/}\VariableTok{$\{sample\}}\StringTok{/k141\_7744.fa"}

\CommentTok{\# Blast k141\_9107}
\BuiltInTok{echo} \StringTok{"{-}{-}{-}{-}{-}{-}{-}{-}{-}{-}{-}{-}{-}{-}{-}{-}{-}{-}{-}{-}{-}{-}{-}{-}{-}{-}{-}{-}{-}{-}"}
\BuiltInTok{echo} \StringTok{"Processing contig: k141\_9107"}
\BuiltInTok{echo} \StringTok{"{-}{-}{-}{-}{-}{-}{-}{-}{-}{-}{-}{-}{-}{-}{-}{-}{-}{-}{-}{-}{-}{-}{-}{-}{-}{-}{-}{-}{-}{-}{-}"}
\VariableTok{$\{envDir\}}\ExtensionTok{/blastn} \AttributeTok{{-}db} \VariableTok{$\{dataDir\}}\NormalTok{/databases/ref\_prok\_rep\_genomes }\AttributeTok{{-}query} \VariableTok{$\{dataDir\}}\NormalTok{/megahit/}\VariableTok{$\{sample\}}\NormalTok{/k141\_9107.fa }\AttributeTok{{-}outfmt} \StringTok{"6 qseqid pident evalue length stitle"} \KeywordTok{|} \FunctionTok{awk} \StringTok{\textquotesingle{}$2 \textgreater{}= 99.0 \&\& $4 \textgreater{}= 1000\textquotesingle{}}

\CommentTok{\# Blast k141\_6789}
\BuiltInTok{echo} \StringTok{"{-}{-}{-}{-}{-}{-}{-}{-}{-}{-}{-}{-}{-}{-}{-}{-}{-}{-}{-}{-}{-}{-}{-}{-}{-}{-}{-}{-}{-}{-}"}
\BuiltInTok{echo} \StringTok{"Processing contig: k141\_6789"}
\BuiltInTok{echo} \StringTok{"{-}{-}{-}{-}{-}{-}{-}{-}{-}{-}{-}{-}{-}{-}{-}{-}{-}{-}{-}{-}{-}{-}{-}{-}{-}{-}{-}{-}{-}{-}{-}"}
\VariableTok{$\{envDir\}}\ExtensionTok{/blastn} \AttributeTok{{-}db} \VariableTok{$\{dataDir\}}\NormalTok{/databases/ref\_prok\_rep\_genomes }\AttributeTok{{-}query} \VariableTok{$\{dataDir\}}\NormalTok{/megahit/}\VariableTok{$\{sample\}}\NormalTok{/k141\_6789.fa }\AttributeTok{{-}outfmt} \StringTok{"6 qseqid pident evalue length stitle"} \KeywordTok{|} \FunctionTok{awk} \StringTok{\textquotesingle{}$2 \textgreater{}= 99.0 \&\& $4 \textgreater{}= 1000\textquotesingle{}}

\CommentTok{\# Blast k141\_7744}
\BuiltInTok{echo} \StringTok{"{-}{-}{-}{-}{-}{-}{-}{-}{-}{-}{-}{-}{-}{-}{-}{-}{-}{-}{-}{-}{-}{-}{-}{-}{-}{-}{-}{-}{-}{-}"}
\BuiltInTok{echo} \StringTok{"Processing contig: k141\_7744"}
\BuiltInTok{echo} \StringTok{"{-}{-}{-}{-}{-}{-}{-}{-}{-}{-}{-}{-}{-}{-}{-}{-}{-}{-}{-}{-}{-}{-}{-}{-}{-}{-}{-}{-}{-}{-}{-}"}
\VariableTok{$\{envDir\}}\ExtensionTok{/blastn} \AttributeTok{{-}db} \VariableTok{$\{dataDir\}}\NormalTok{/databases/ref\_prok\_rep\_genomes }\AttributeTok{{-}query} \VariableTok{$\{dataDir\}}\NormalTok{/megahit/}\VariableTok{$\{sample\}}\NormalTok{/k141\_7744.fa }\AttributeTok{{-}outfmt} \StringTok{"6 qseqid pident evalue length stitle"} \KeywordTok{|} \FunctionTok{awk} \StringTok{\textquotesingle{}$2 \textgreater{}= 99.0 \&\& $4 \textgreater{}= 1000\textquotesingle{}}
\end{Highlighting}
\end{Shaded}

\begin{verbatim}
------------------------------
Processing contig: k141_9107
-------------------------------
------------------------------
Processing contig: k141_6789
-------------------------------
------------------------------
Processing contig: k141_7744
-------------------------------
\end{verbatim}

}

\end{table}%

The \texttt{BLAST} results indicate a statistically definitive match
between the query sequence and the subject database. The alignment
parameters - specifically the 99.409\% percent identity, the 5757
basepair alignmnent length, and the E-value of 0.0 - provides evidence
of high-confidence sequence homology.

\begin{itemize}
\item
  \texttt{Percent\ Identity}: The high degree of nucleotide conservation
  suggests that the query and subject sequences are nearly identical.
  This level of similarity is characteristic of sequences derived from
  the same species or highly conserved orthologous rehions within
  closely related strains.
\item
  \texttt{Alignment\ Length}: The substantial length of the aligned
  region bolsters the validity of the match.
\item
  \texttt{Expect\ Value}: An E-value of 0.0 represents a probability
  that is indistinguishable from zero. It signifies that the likelyhood
  of observing such a high-scoring alignment by random chance within the
  search space is non-existent.
\end{itemize}

The verified non-host reads were assembled into contigs using
\texttt{megahit} (Li et al., 2016) and subsequently classified via
\texttt{BLAST} (Camacho et al., 2009). The analysis yielded a
high-confidence identification based on the alignment of contig
\texttt{k141\_9107}

\begin{table}

\caption{\label{tbl-amrfinder}AMRFinder Results for Sample}

\centering{

\begin{Shaded}
\begin{Highlighting}[]
\VariableTok{envDir}\OperatorTok{=}\StringTok{"/home/devonallies/micromamba/envs/bioinfo/bin"}
\VariableTok{dataDir}\OperatorTok{=}\StringTok{"../../../../data"}
\VariableTok{sample}\OperatorTok{=}\VariableTok{$(}\FunctionTok{basename} \StringTok{"}\VariableTok{$\{dataDir\}}\StringTok{/unmapped/SRR35359600"}\NormalTok{ \_R1.fq}\VariableTok{)}

\VariableTok{$\{envDir\}}\ExtensionTok{/amrfinder} \AttributeTok{{-}n} \VariableTok{$\{dataDir\}}\NormalTok{/megahit/}\VariableTok{$\{sample\}}\NormalTok{/}\VariableTok{$\{sample\}}\NormalTok{.contigs.fa }\AttributeTok{{-}{-}database} \VariableTok{$\{dataDir\}}\NormalTok{/databases/armfinder/amr\_db/2026{-}01{-}21.1/ }\AttributeTok{{-}{-}plus}
\end{Highlighting}
\end{Shaded}

\begin{verbatim}
Running: /home/devonallies/micromamba/envs/bioinfo/bin/amrfinder -n ../../../../data/megahit/SRR35359600/SRR35359600.contigs.fa --database ../../../../data/databases/armfinder/amr_db/2026-01-21.1/ --plus
Software directory: /home/devonallies/micromamba/envs/bioinfo/bin/
Software version: 4.2.5
Database directory: /mnt/069ABB979ABB822B/learning/data/databases/armfinder/amr_db/2026-01-21.1
Database version: 2026-01-21.1
AMRFinder translated nucleotide search
  - include -O ORGANISM, --organism ORGANISM option to add mutation searches and suppress common proteins
Running blastx
Making report
Protein id  Contig id   Start   Stop    Strand  Element symbol  Element name    Scope   Type    Subtype Class   Subclass    Method  Target length   Reference sequence length   % Coverage of reference % Identity to reference Alignment length    Closest reference accession Closest reference name  HMM accession   HMM description
NA  k141_2874   760 1617    +   blaTEM-116  broad-spectrum class A beta-lactamase TEM-116   core    AMR AMR BETA-LACTAM BETA-LACTAM ALLELEX 286 286 100.00  100.00  286 WP_000027050.1  broad-spectrum class A beta-lactamase TEM-116   NA  NA
amrfinder took 45 seconds to complete
\end{verbatim}

}

\end{table}%

\texttt{AMRFINDER} identified the gene \texttt{blaTEM-116}, which is a
broad-spectrum class A beta-lactamase. This enzyme breaks down
beta-lactam antibiotics.

\subsection{Conclusion:}\label{conclusion}

The identification of \emph{Acinetobacter parvus}, is a gram-negative
aerobic bacterium notable for forming small colonies in cultures, with
the metrics displayed in Table~\ref{tbl-idsample}, indicates a strong
result.

\subsection*{Reference:}\label{reference}
\addcontentsline{toc}{subsection}{Reference:}

\phantomsection\label{refs}
\begin{CSLReferences}{0}{1}
\bibitem[\citeproctext]{ref-blast}
Camacho, C., Coulouris, G., Avagyan, V., Ma, N., Papadopoulos, J.,
Bealer, K. and Madden, T.L., 2009. BLAST+: Architecture and
applications. \emph{BMC Bioinformatics}, 10(1), p.421.
https://doi.org/\href{https://doi.org/10.1186/1471-2105-10-421}{10.1186/1471-2105-10-421}.

\bibitem[\citeproctext]{ref-Chen2025}
Chen, G., Wiegand, C., Willett, A., Herr, C., Müller, R., Bals, R. and
Kalinina, O.V., 2025. Comparative metagenomic analysis on {COPD} and
health control samples reveals taxonomic and functional motifs.
\emph{Frontiers in Microbiology}, 16, p.1636322.
https://doi.org/\href{https://doi.org/10.3389/fmicb.2025.1636322}{10.3389/fmicb.2025.1636322}.

\bibitem[\citeproctext]{ref-megahit}
Li, D., Luo, R., Liu, C.-M., Leung, C.-M., Ting, H.-F., Sadakane, K.,
Yamashita, H. and Lam, T.-W., 2016. {MEGAHIT} v1.0: A fast and scalable
metagenome assembler driven by advanced methodologies and community
practices. \emph{Methods}, 102, pp.3--11.
https://doi.org/\href{https://doi.org/10.1016/j.ymeth.2016.02.020}{10.1016/j.ymeth.2016.02.020}.

\bibitem[\citeproctext]{ref-bwa}
Li, H. and Durbin, R., 2009. Fast and accurate short read alignment with
burrows-wheeler transform. \emph{Bioinformatics}, 25(14), pp.1754--1760.
https://doi.org/\href{https://doi.org/10.1093/bioinformatics/btp324}{10.1093/bioinformatics/btp324}.

\bibitem[\citeproctext]{ref-seqkit}
Shen, W., Sipos, B. and Zhao, L., 2024. SeqKit2: A swiss army knife for
sequence and alignment processing. \emph{iMeta}, {[}online{]} 3(3),
p.e191.
https://doi.org/\href{https://doi.org/10.1002/imt2.191}{10.1002/imt2.191}.

\end{CSLReferences}




\end{document}
